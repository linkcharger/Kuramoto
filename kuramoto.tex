\documentclass[10pt,a4paper]{article}
\usepackage[utf8]{inputenc}
\usepackage[english]{babel}
\usepackage{graphicx}
\usepackage{lmodern}
\usepackage[
	margin		= 20mm,
	top 		= 22mm,
	bottom 		= 10mm, 
	footnotesep = 2\baselineskip	
]{geometry}			
%\usepackage{listings}

\usepackage{fancyhdr}
\pagestyle{fancy}
\fancyhf{}																	% clears all headers and footers
\fancyhead[c]{}
\fancyhead[l]{Aaron Schade}
\fancyhead[r]{\thepage}
\renewcommand{\headrulewidth}{1pt}
%\renewcommand{\footrulewidth}{1pt}

%\setlength{\headsep}{5mm}


%%%%%%%%%%%%%%%%%%%%%%%%%%%%%%%%%%%%%%%%%%%%%%%%%%%%%%%%%%%%%%%%%%%%%%%%%%%%%%%%%%%%%%%%%%%%%%%%%%%%%%%%%%%%%%%%%%%%%%%%%%%%%%%%%%%%%%%%%%%%%%%%%%%%%%%%%%%%%%%%%%%%%%%%%%%%%%%%%%%%%%%%%%%%%%%%%%%%%%%%


\usepackage[
	notes,
%	short,										% even first mention is shortened
	genallnames,								% when using \gentextcite for possessive cases, use all authors names
%	strict,										% erases the line before footnotes?
	urlnotes		= false,					% disable url, doi, eprint in notes but not in bibliography; alternatively: includeall  false OR doi % false, isbn % false, url % false,	eprint % false
	eprint			= false,
	backend 		= biber,					% biber's more advanced than bibtex
	autolang 		= other,
%	bibencoding 	= utf8,						% before: latin1 -> fucked up everything, couldnt recognise utf8 characters (duh!)
	compresspages]								% something like 321--328 in your .bib file would become 321–28
{biblatex-chicago}
\addbibresource{thesis.bib}

\AtEveryCitekey{\clearfield{isbn}}				% show isbn in bib, as before, but clear in citations
\AtEveryCitekey{\clearfield{issn}}	
\AtEveryCitekey{\clearfield{note}}	


\AtEveryCitekey{\clearfield{series}}	
\AtEveryCitekey{\clearlist{publisher}}	
\AtEveryCitekey{\clearlist{location}}	
\AtEveryCitekey{\clearfield{journaltitle}}	
\AtEveryCitekey{\clearfield{volume}}	
\AtEveryCitekey{\clearfield{pages}}	
\AtEveryCitekey{\clearfield{edition}}	
\AtEveryCitekey{\clearfield{date}}	
\AtEveryCitekey{\clearfield{year}}	
\AtEveryCitekey{\clearfield{institution}}	
\AtEveryCitekey{\clearfield{number}}	
\AtEveryCitekey{\clearfield{booktitle}}	
%\AtEveryCitekey{\clearname{editor}}	


\renewcommand*{\bibfont}{\small}
\DeclareDelimFormat{finalnamedelim}{\addspace\bibstring{and}\space}			% remove oxford comma



\usepackage[multiple, hang]{footmisc}
\setlength{\footnotemargin}{3mm}



%\usepackage{fnpct}											% improves kerning (spacing) of footnotemakrs above punctuation, could potentially do 'multiple' but doesnt work for citations (or at least its annoying with footnote citations)



%%%%%%%%%%%%%%%%%%%%%%%%%%%%%%%%%%%%%%%%%%%%%%%%%%%%%%%%%%%%%%%%%%%%%%%%%%%%%%%%%%%%%%%%%%%%%%%%%%%%%%%%%%%%%%%%%%%%%%%%%%%%%%%%%%%%%%%%%%%%%%%%%%%%%%%%%%%%%%%%%%%%%%%%%%%%%%%%%%%%%%%%%%%%%%%%%%%%%%%%



% important to load AFTER packages affecting referencing of any kind
\usepackage[
	linktoc			= all,								
	hyperfootnotes	= false, 				% footnotemarks are hyper to footnotetext - easily broken :/
%	hyperindex,								% numbers in index are hyper
	hidelinks, 								% just disables colour and border
%	bookmarksopen 	= false					% doesnt work to hide bookmarks, instead:
	pdfpagemode		= UseNone, 
	pdftitle 		= 'glorious title'
]{hyperref}




%%%%%%%%% misc %%%%%%%%%%%%%%%%%%%%%%%%%%%%%%%%%%%%%%%%%%%%%%%%%%%%%%%%%%%%%%%%%%%%%%%%%%%%%%%%%%%%%%%%%%%%%%%%%%%%%%%%%%%%%%%%%%%%%%%%%%%%%%%%%%%%%%%%%%%%%%%%%%%%%%%%%%%%%%%%%%%%%%%%%%%%%%%%%%%%%%%%%

%\usepackage{lipsum}


\usepackage{enumitem}
	\setlist{nosep}
%\usepackage{setspace}
%\usepackage{sectsty}
%\allsectionsfont{\centering}


%\usepackage{multicol}
%\newcommand{\fixspacing}{\vspace{0pt plus 1filll}\mbox{}}		% makes paragraphs not stretch in multicol - maybe?


\newcommand{\osc}{\texttt{Oscillator}~}
\newcommand{\oscpop}{\texttt{OscPopulation}~}


\newcommand{\graph}{\medskip\noindent}





%%%%%%%%%%%%%%%%%%%%%%%%%%%%%%%%%%%%%%%%%%%%%%%%%%%%%%%%%%%%%%%%%%%%%%%%%%%%%%%%%%%%%%%%%%%%%%%%%%%%%%%%%%%%%%%%%%%%%%%%%%%%%%%%%%%%%%%%%%%%%%%%%%%%%%%%%%%%%%%%%%%%%%%%%%%%%%%%%%%%%%%%%%%%%%%%%%%%%%%%
%%%%%%%%%%%%%%%%%%%%%%%%%%%%%%%%%%%%%%%%%%% to do %%%%%%%%%%%%%%%%%%%%%%%%%%%%%%%%%%%%%%%%%%%%%%%%%%%%%%%%%%%%%%%%%%%%%%%%%%%%%%%%%%%%%%%%%%%%%%%%%%%%%%%%%%%%%%%%%%%%%%%%%%%%%%%%%%%%%%%%%%%%%%%%%%%%%%
















%%%%%%%%%%%%%%%%%%%%%%%%%%%%%%%%%%%%%%%%%%%%%%%%%%%%%%%%%%%%%%%%%%%%%%%%%%%%%%%%%%%%%%%%%%%%%%%%%%%%%%%%%%%%%%%%%%%%%%%%%%%%%%%%%%%%%%%%%%%%%%%%%%%%%%%%%%%%%%%%%%%%%%%%%%%%%%%%%%%%%%%%%%%%%%%%%%%%%%%%
%%%%%%%%%%%%%%%%%%%%%%%%%%%%%%%%%%%%%%%%%%%%%%%%%%%%%%%%%%%%%%%%%%%%%%%%%%%%%%%%%%%%%%%%%%%%%%%%%%%%%%%%%%%%%%%%%%%%%%%%%%%%%%%%%%%%%%%%%%%%%%%%%%%%%%%%%%%%%%%%%%%%%%%%%%%%%%%%%%%%%%%%%%%%%%%%%%%%%%%%







\begin{document}

%%%%%%%%%%%%%%%%%%%%%%%%%%%%%%%%%%%%%%%%%%%%%%%%%%%%%%%%%%%%%%%%%%%%%%%%%%%%%%%%%%%%%%%%%%%%%%%%%%%%%%%%%%%%%%%%%%%%%%%%%%%%%%%%%%%%%%%%%%%%%%%%%%%%%%%%%%%%%%%%%%%%%%%%%%%%%%%%%%%%%%%%%%%%%%%%%%%%%%%%
\section{Theoretical study}
















%%%%%%%%%%%%%%%%%%%%%%%%%%%%%%%%%%%%%%%%%%%%%%%%%%%%%%%%%%%%%%%%%%%%%%%%%%%%%%%%%%%%%%%%%%%%%%%%%%%%%%%%%%%%%%%%%%%%%%%%%%%%%%%%%%%%%%%%%%%%%%%%%%%%%%%%%%%%%%%%%%%%%%%%%%%%%%%%%%%%%%%%%%%%%%%%%%%%%%%%
\section{Numerical study}


{\large\textbf{Omegas -- normally distributed}}
%%%%%%%%%%%%%%%%%%%%%%%%%%%%%%%%%%%%%%%%%%%%%%%%%%%%%%%%%%%%%%%%%%%%%%%%%%%%%%%%%%%%%%%%%%%%%%%%%%%%%%%%%%%%%%%%%%%%%%%%%%%%%%%%%
\subsection{$r_{\infty}(K)$ -- simulated vs predicted}


\subsubsection{Euler method}

\paragraph{Design and initiation}
My implementation of the Euler method is object-oriented. 
This may not be the computationally most efficient way, but I'm still somewhat of a beginner with code, so I appreciate the clarity that object-oriented programming affords.

The first type of object is simply an \osc with three variables: $\omega, \theta_{t-1}$ and $\theta_t$. 
A $\Delta t = 1$ is enough in the Euler method.
An \osc gets initiated with its natural frequency and the current-period $\theta$. 
The last-period $\theta$ is initiated as zero, since the first part of making an 'Euler step' forwards is to hand over $\theta_{t}$ to $\theta_{t-1}$. 

The other object type is the \oscpop. 
It consists of a list of \texttt{Oscillators} and a construction mode, namely the distribution according to which the $\omega$s are distributed.
When an \oscpop is initiated, for each \osc a natural frequency and initial phase are drawn from the respective type of random distribution. 
An \osc object is then initiated with those values and assigned to a place in the list within the \oscpop object.


\paragraph{Running the simulation}











\subsubsection{Numerical integration}


The strategy to integrate the consistency equation is to loop through different values of $r$ and check which ones make the right-hand side approximately equal to one. 
In a loop around this one, we cycle through the different values for K to get the values of the desired $r_\infty(K)$ relationship. 















%%%%%%%%%%%%%%%%%%%%%%%%%%%%%%%%%%%%%%%%%%%%%%%%%%%%%%%%%%%%%%%%%%%%%%%%%%%%%%%%%%%%%%%%%%%%%%%%%%%%%%%%%%%%%%%%%%%%%%%%%%%%%%%%%
\subsection{$r(t)$ for $K = [1, 2]$}












{\large\textbf{Omegas -- uniformly distributed}}
%%%%%%%%%%%%%%%%%%%%%%%%%%%%%%%%%%%%%%%%%%%%%%%%%%%%%%%%%%%%%%%%%%%%%%%%%%%%%%%%%%%%%%%%%%%%%%%%%%%%%%%%%%%%%%%%%%%%%%%%%%%%%%%%%
\subsection{$r_{\infty}(K)$}















%%%%%%%%%%%%%%%%%%%%%%%%%%%%%%%%%%%%%%%%%%%%%%%%%%%%%%%%%%%%%%%%%%%%%%%%%%%%%%%%%%%%%%%%%%%%%%%%%%%%%%%%%%%%%%%%%%%%%%%%%%%%%%%%%
\subsection{$r(t)$ for different initial conditions $\theta_0$}






















%%%%%%%%%%%%%%%%%%%%%%%%%%%%%%%%%%%%%%%%%%%%%%%%%%%%%%%%%%%%%%%%%%%%%%%%%%%%%%%%%%%%%%%%%%%%%%%%%%%%%%%%%%%%%%%%%%%%%%%%%%%%%%%%%
\subsection{$r(t)$ for different initial conditions $\omega_0$}





































\end{document}